\documentclass[10pt,a4paper]{article}

\newcommand{\COLORSDIR}{/Users/hoolywear/Desktop/UNIMORE/II ANNO/II SEMESTRE/colors}

\usepackage[italian]{babel}
\usepackage[usenames,dvipsnames,table]{xcolor}
\usepackage[utf8]{inputenc}
\usepackage[T1]{fontenc}
\usepackage{soul}
\usepackage[a4paper, portrait, margin=2.5cm]{geometry}
\usepackage{array}
\usepackage{tabularx}
\usepackage{multicol}
\usepackage{amsmath}
\usepackage{amsfonts}
\usepackage{amssymb}
\usepackage{algorithmicx}
\usepackage[noend]{algpseudocode}
\usepackage{wrapfig}
\usepackage{graphicx}
\usepackage{hyperref}
\hypersetup{
    colorlinks=true,
    linkcolor=black,
    filecolor=magenta,      
    urlcolor=cyan,
    pdftitle={Overleaf Example},
    pdfpagemode=FullScreen,
    }
\urlstyle{same}
\usepackage{caption}
\usepackage{capt-of}
\captionsetup[figure]{name=Fig.}
\renewcommand{\thefigure}{\arabic{section}.\arabic{figure}}
\graphicspath{ {./images/} }

\input{\COLORSDIR/colors_4}

\usepackage{listings}

\definecolor{codeblue}{HTML}{1e66f5}
\definecolor{codepurple}{HTML}{8839ef}
\definecolor{codered}{HTML}{d20f39}
\definecolor{darkbluenord}{HTML}{232731}
\definecolor{lightbluenord}{HTML}{b1bfe3}

\lstdefinestyle{code}{
    backgroundcolor=\color{gray!10},   
    basicstyle=\ttfamily,
    commentstyle=\color{codeblue},
    keywordstyle=\color{codered},
    breakatwhitespace=false,         
    breaklines=true,                 
    captionpos=b,                    
    keepspaces=true,                 
    showspaces=false,                
    showstringspaces=false,
    showtabs=false,                  
    tabsize=2,
    mathescape=true %dollar signs act as inline math delimiters
}

\lstdefinestyle{python}{
    style=code,
    language=python,
    commentstyle=\color{codeblue},
    keywordstyle=\color{codepurple},
    stringstyle=\color{codered}
}

\lstset{style=code,language=C++}

\usepackage[framemethod=TikZ]{mdframed}

\mdfsetup{%
  roundcorner=8pt}

% styles
\def\Clinewidth{.8pt}
\mdfdefinestyle{titlerule}{%
  frametitlerule=true,%
  frametitlerulewidth=\Clinewidth,%
  subtitleaboveline=true,subtitlebelowline=true,%
  subtitleabovelinewidth=\Clinewidth,subtitlebelowlinewidth=\Clinewidth,%
linewidth=1pt}

\mdfdefinestyle{emphasize}{%
  style=titlerule,%
  frametitle=,%
  linecolor=gray!50,linewidth=1pt,backgroundcolor=gray!10}

% algorithmic environment
\surroundwithmdframed[backgroundcolor=gray!10,hidealllines=true,%
frametitle={}]{algorithmic}

% quote environment
\surroundwithmdframed[style=emphasize]{quote}

% example environment
\newmdenv[frametitle=Esempio,style=titlerule]{example}

% emphasize environment
\newmdenv[style=emphasize,%
          linecolor=emp!70!red,backgroundcolor=emp]{emphasize}

% blue emphasize environment
\newmdenv[style=emphasize,%
          linecolor=obs!70,backgroundcolor=obs!20]{emphasize-blue}

%% NOT USED IN III ANNO
%% % definition environment
%% \newmdenv[frametitle=Definizione,style=titlerule,%
%%           linecolor=def]{definition}
%% 
%% % theorem environment
%% \newmdenv[frametitle=Teorema,style=titlerule,%
%%           linecolor=the]{theorem}
%% % observation environment
%% \newmdenv[frametitle=Osservazione,%
%%           backgroundcolor=white,linecolor=obs,%
%%           frametitlebackgroundcolor=obs]{observation}
%% 
%% % warning environment
%% \newmdenv[style=emphasize,%
%%           backgroundcolor=war!10,linecolor=war]{warning}

\author{Iacopo Ruzzier}
\date{Ultimo aggiornamento: \today}


\title{%
  Progetto del Software\\
\large Appunti redatti}

\begin{document}
\maketitle
\tableofcontents
\newpage

\section*{Introduzione}
\addcontentsline{toc}{section}{Introduzione}

Il corso tratta metodologie di progetto del software, diversificate per le varie necessit\`a e solitamente riconducibili a poche famiglie $\rightarrow$ strutturazione e automatizzazione del processo

vedremo
\begin{itemize}
  \item documentazione (su sw di larga scala si lavora in molti) - come usare, modificare, testare, rilasciare i nostri artefatti sw; licensing
  \item requisiti, KPI, test-driven design
  \item tool di collaborazione
  \item design tools (UML, E/R d., ...)
  \item design pattern
  \item strutturazione codice (programmazione avanzata)
  \item alcuni use-cases
\end{itemize}

\newpage
\part{Collaborative tools - Git \& friends}
\section{VCS, git}

vcs: sistema che traccia le modifiche fatte ad un progetto e permette di ritornare a stati precedenti
\begin{itemize}
  \item permette lo sviluppo collaborativo (modifiche "firmate")
  \item obbliga a seguire flussi di sviluppo noti
  \item fornisce set di strumenti per automatizzare testing, integrazione, sviluppo (CI/CD)
  \item modo semplice per scrivere documentazione
  \item permette di concentrarsi sulla scrittura del codice "senza pensieri"
\end{itemize}

\subsection{git}

vcs pi\`u usato, basato su repo distribuiti (2005 torvalds, per supportare sviluppo kernel linux): a partire da repo remote contenenti la codebase ("origin") gli utenti clonano la repo in locale, la mantengono aggiornata tramite pull, la aggiornano tramite push

\section{Workflow base}

progetto come \textbf{sequenza di commit}: snapshot del codice in un dato istante, con commit identificati da hashcode, contenenti riferimento al commit precedente, e commento obbligatorio\\
$\rightarrow$ i commit tracciano cambiamenti incrementali della codebase, con granularit\`a a discrezione dei programmatori

\begin{emphasize}
    \begin{itemize}
      \item prima di un commit devo aggiungere i file nella staging area
      \item rinominare file significa elimino+riaggiungo
    \end{itemize}
\end{emphasize}

\begin{lstlisting}[language=bash]
git clone <URL>
git log #log di tutti i commit fatti
git diff #per visualizzare le differenze tra due commit
git add <file>[<file>...] #staging
git commit # -m "msg"
git commit -a # bypassa staging, ma non considera i file appena aggiunti
			        # non ancora tracciati
git commit --amend # ritorno al commit precedente se ho dimenticato qualcosa
git push [origin] [master]
git pull [origin] [master]\end{lstlisting}

IMG schema git

\begin{emphasize}
    pulling e auto-merging: la storia locale \`e aggiornata in base al timestamp del commit. in particolare viene modificata automaticamente includendo sia i cambiamenti locali che quelli remoti (merge in automatico)
\end{emphasize}

\subsection{Merge e conflitti}

git lavora sulla singola riga $\rightarrow$ conflitto se viene modificata da pi\`u utenti diversi, con repo locale che resta in stato conflicting in caso di \textbf{merge conflicts} $\rightarrow$ vanno risolti localmente e va fatto il merge manuale (flag appropriati)

consigli:
\begin{itemize}
  \item pull frequenti
  \item assicurarsi che il codice funzioni (testing automations)
  \item commit piccoli, suddivisi per area di lavoro (codice, makefile, script) $\rightarrow$ forza a mantenere lo spazio di lavoro pulito e strutturato
\end{itemize}

\begin{lstlisting}[language=bash]
git checkout/reset # unstaging/deletion modifiche o commit locali
git revert # per ritornare ad un commit specifico
git cherry-pick # essendo i commit incrementali (tracciano la
# differenza rispetto al genitore), possiamo applicare la stessa
# logica ma rispetto ad un commit diverso\end{lstlisting}

\section{Struttura tipica di un progetto con versioning}

tipicamente molto rigida
\begin{itemize}
  \item branch principale contenente ultima versione rilasciata (e intera cronologia commit)
  \item branch multiple corrispondenti a sottoprogetti specifici
  \item libert\`a totale sulle branch, tipicam.~regole aziendali (\lstinline|develop,bugfix/,hotfix/,features/,.pb_<smth>|)
  \item push su main branch non permesso $\rightarrow$ fork di main o dev, e poi PR (gestita dal maintainer)
  \item regole di accesso e vari ruoli utente (a liv.~repo e branch)
\end{itemize}

flow tipico:
\begin{enumerate}
  \item dev clona una branch della repo remota aggiornata
  \item dev inizia a lavorare, nuovi commit in lcoale, nel mentre commit nuovi anche in remoto
  \item una volta pronto, il dev fa una pull da remoto, con la responsabilit\`a di rendere consistenti i propri commit con la cronologia principale (implica retesting)
  \item dopo il merge, si crea il "final commit", si pusha sul cloud e si fa una PR
  \item (tipicamente) la richiesta viene accettata, e le modifiche sono applicate alla branch \lstinline|Developer|
  \item ultimo pull per rendere consistente la copia locale, e (tipicamente) eliminiamo l'altra branch
\end{enumerate}

\newpage
\part{The software design process}

\newpage
\part{Requisiti e specifiche}

\newpage
\part{Documentazione - Notazioni e strumenti}

\newpage
\part{UML}

\newpage
\part{System architecture and design}

\section{System design}

Step in cui traduciamo le specifiche del cliente in specifiche tecnologiche per gli sviluppatori $\rightarrow$ output: \textbf{architettura del sistema}
\begin{itemize}
  \item identifichiamo un insieme di moduli, ciascuno con una singola funzionalit\`a specifica
  \item descriviamo i \textit{contratti}, le interazioni tra di loro
\end{itemize}

\subsection{Decomposizione in moduli}

\begin{itemize}
  \item decomponiamo prima in \textbf{sottosistemi}, che interagiscono ma \textbf{non dipendono} tra loro
  \item poi in moduli e sottomoduli, ciascuno con il loro servizio specifico
  \item poi in componenti (unit\`a implementativa di base)
\end{itemize}

La scelta alla base della decomposizione sta ovviamente nella separazione dei servizi offerti, e nell'assegnamento del controllo (chi controlla cosa)

\subsection{Moduli}
\begin{itemize}
  \item si raggruppano funzionalit\`a in stretta relazione (es. CRUD relative agli account utente) $\rightarrow$ a livello di system design dobbiamo specificare chiaramente le \textbf{interfacce} verso altri moduli o l'esterno
  \item successivamente identifico le sotto-funzionalit\`a
\end{itemize}

\subsubsection{Relazioni tra moduli}

Tipicamente i moduli espongono servizi usati da altri, sono composti da sottomoduli (per lavorare a livelli diversi di dettaglio) e possono dipendere da altri moduli (tipicamente seguendo dei diagrammi di sequenza per use-case specifici)

\subsubsection{Strategia di partizionamento}

Top down:
\begin{itemize}
  \item parto dalle specifiche e dalla documentazione
  \item arrivo ai servizi, moduli, componenti eccetera
\end{itemize}

Bottom up:
\begin{itemize}
  \item data-structure/functionality centric $\rightarrow$ quando ad es. parto da sistemi preesistenti
\end{itemize}

L'approccio reale \`e ovviamente ibrido

\section{Pattern architetturali}

\subsection{Architettura client-server}

Tipico di sistemi distribuiti, composto di 
\begin{itemize}
  \item 1+ server che offrono generici servizi
  \item + client che usano i servizi
  \item 1 network di comunicazione, che si presume sempre attivo
\end{itemize}

Comnunicazione asimmetrica (requests e responses)

Pro:
\begin{itemize}
  \item facile distribuire dati e responsabilit\`a ?
  \item scalabilit\`a in termini di client e di server
\end{itemize}

Contro:
\begin{itemize}
  \item la scalabilit\`a "costa" (vado a scalare semplicemente aumentando le risorse)
  \item bisogno di un naming service: i server devono essere conosciuti dai client!
  \item evidente situazione di dipendenza
\end{itemize}

\subsection{Architettura multi-tier}

immagine

\subsection{Design del controllo}

Controllo inteso come "chi fa cosa", ovvero chi possiede la logica che implementa i casi d'uso $\rightarrow$ centralizzato o decentralizzato, con chiare conseguenze sull'architettura del sistema

\subsubsection{Controllo centralizzato (sincrono)}

Un singolo sistema (es. server web) gestisce (serves) tutte le richieste
\begin{itemize}
  \item dipende da altri sottosistemi
  \item tipico per frontend di web-app
  \item basato su comunicazioni sincrone (es. chiamate di funzione)
\end{itemize}

I pro e contro direttamente derivati sono \textbf{single point of access} e \textbf{single point of failure}

\paragraph{Master-Slave}~\\

Si presenta nei paradigmi multi-process e multi-thread, in ogni caso richiede comunicazione tra processi

\subsubsection{Controllo event-based (asincrono)}

Ciascun modulo o sottosistema lavora in maniera indipendente dagli altri, e si relaziona con l'esterno tramite comunicazioni asincrone\\
$\rightarrow$ pro e contro: sistema distribuito, dunque pi\`u difficile da implementare ma con meno dipendenze tra moduli

Paradigma event-based: il message broker \`e il first-class citizen, ovvero il modulo al quale si interfacciano tutti gli altri e che gestisce tutte le comunicazioni $\rightarrow$ si sposa benissimo con un frontend che deve mandare richieste (e poi si mette in attesa) - se async non ha bisogno di tenere occupate le risorse del sistema

\subsection{Specifiche per il singolo modulo}

Una volta definito il modello architetturale, si decompongono le macroaree nei singoli moduli $\rightarrow$ comincio a decidere dettagli implementativi (es.~dove si trovano le funzionalit\`a, su che server...), seguendo alcuni principi base:
\begin{itemize}
  \item loose-coupling: maggiore indipendenza possibile
  \item minima conoscenza inter-modulare tra sviluppatori
  \item high cohesion: i moduli raggruppano le funzionalit\`a strettamente dipendenti
\end{itemize}

Per prima cosa dobbiamo definire i contratti (interfacce) tra i vari moduli (professionalmente si usa UML):
\begin{itemize}
  \item che funzionalit\`a espongo: es.~update dell'et\`a, rimozione utente...
  \item come le espongo: servizi o funzioni da chiamare, es.~REST API
  \item parametri input-output: numero, tipo, eccetera
\end{itemize}

\section{MVC - Model View Controller pattern}



\end{document}
