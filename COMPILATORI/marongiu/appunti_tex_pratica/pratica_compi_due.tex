\documentclass[10pt,a4paper]{article}

\newcommand{\COLORSDIR}{/Users/hoolywear/Desktop/UNIMORE/II ANNO/II SEMESTRE/colors}

\usepackage[italian]{babel}
\usepackage[usenames,dvipsnames]{xcolor}
\usepackage[utf8]{inputenc}
\usepackage[T1]{fontenc}
\usepackage{soul}
\usepackage[a4paper, portrait, margin=2.5cm]{geometry}
\usepackage{array}
\usepackage{tabularx}
\usepackage{multicol}
\usepackage{amsmath}
\usepackage{amsfonts}
\usepackage{amssymb}
\usepackage{algorithmicx}
\usepackage[noend]{algpseudocode}
\usepackage{wrapfig}
\usepackage{graphicx}
\usepackage{hyperref}
\hypersetup{
    colorlinks=true,
    linkcolor=black,
    filecolor=magenta,      
    urlcolor=cyan,
    pdftitle={Overleaf Example},
    pdfpagemode=FullScreen,
    }

\urlstyle{same}
\graphicspath{ {./images/} }

\definecolor{emp}{HTML}{f9e9ec}
\definecolor{war}{HTML}{f88dad}
\definecolor{def}{HTML}{fac748}
\definecolor{the}{HTML}{1d2f6f}
\definecolor{obs}{HTML}{8390fa}


\usepackage{listings}

\definecolor{codeblue}{HTML}{1e66f5}
\definecolor{codepurple}{HTML}{8839ef}
\definecolor{codered}{HTML}{d20f39}
\definecolor{darkbluenord}{HTML}{232731}
\definecolor{lightbluenord}{HTML}{b1bfe3}

\lstdefinestyle{code}{
    backgroundcolor=\color{gray!10},   
    basicstyle=\ttfamily,
    commentstyle=\color{codeblue},
    keywordstyle=\color{codered},
    breakatwhitespace=false,         
    breaklines=true,                 
    captionpos=b,                    
    keepspaces=true,                 
    showspaces=false,                
    showstringspaces=false,
    showtabs=false,                  
    tabsize=2,
    mathescape=true %dollar signs act as inline math delimiters
}
\lstdefinestyle{sql}{
    style=code,
    language=SQL,
    commentstyle=\color{codeblue},
    keywordstyle=\color{codepurple},
    stringstyle=\color{codered}
}

\lstdefinestyle{python}{
    style=code,
    language=python,
    commentstyle=\color{codeblue},
    keywordstyle=\color{codepurple},
    stringstyle=\color{codered}
}

\lstset{style=code,language=C++}

\usepackage[framemethod=TikZ]{mdframed}

\mdfsetup{%
  roundcorner=8pt}

% styles
\def\Clinewidth{.8pt}
\mdfdefinestyle{titlerule}{%
  frametitlerule=true,%
  frametitlerulewidth=\Clinewidth,%
  subtitleaboveline=true,subtitlebelowline=true,%
  subtitleabovelinewidth=\Clinewidth,subtitlebelowlinewidth=\Clinewidth,%
subtitlebackgroundcolor=obs,linewidth=1pt}
\mdfdefinestyle{emphasize}{%
  style=titlerule,%
  frametitle=,%
  linecolor=gray!50,linewidth=1pt,backgroundcolor=gray!10}

% verbatim environment
\surroundwithmdframed[backgroundcolor=gray!5,hidealllines=true,%
                      innerleftmargin=1pt,innerrightmargin=1pt%
                      frametitle={}]{verbatim}

% algorithmic environment
\surroundwithmdframed[backgroundcolor=gray!10,hidealllines=true,%
frametitle={}]{algorithmic}

% quote environment
\surroundwithmdframed[style=emphasize]{quote}

% example environment
\newmdenv[frametitle=Esempio,style=titlerule]{example}

% definition environment
\newmdenv[frametitle=Definizione,style=titlerule,%
          linecolor=def]{definition}

% theorem environment
\newmdenv[frametitle=Teorema,style=titlerule,%
          linecolor=the]{theorem}

% emphasize environment
\newmdenv[style=emphasize,%
          linecolor=emp!70!red,backgroundcolor=emp]{emphasize}

% observation environment
\newmdenv[frametitle=Osservazione,%
          backgroundcolor=white,linecolor=obs,%
          frametitlebackgroundcolor=obs]{observation}

% warning environment
\newmdenv[style=emphasize,%
          backgroundcolor=war!10,linecolor=war]{warning}

\author{Iacopo Ruzzier}
\date{Ultimo aggiornamento: \today}


\title{%
Compilatori\\
\large Parte Due}

\begin{document}
\maketitle
\tableofcontents
\newpage
\part{lab 1}

\section{la IR di LLVM}

ricordiamo: IR di llvm ha sintassi e semantica simili all'assembly a cui siamo abituati

domanda: come scrivere un passo llvm?

prima chiariamo alcuni punti:
\begin{itemize}
  \item moduli llvm
  \item iteratori
  \item downcasting
  \item interfacce dei passi llvm
\end{itemize}

\subsection{moduli llvm}

un modulo rappresenta un singolo file sorgente (corrisponde) - vedremo che gli iteratori permettono di "scorrere" attraverso tutte le funzioni di un modulo

recupera slide 23 bene



\subsection{iteratori}

nota su cambiamento a nuovo llvm pass manager (vedi link a slide 22)

vediamo che in generale un iteratore permette di puntare al "livello sottostante" della gerarchia appena vista

nota: sintassi simile a quella del container STL \lstinline|vector|

IMG slide 25

caption: l'attraversamento delle strutture dati della IR llvm normalmente avviene tramite doubly-linked lists

\subsection{downcasting}

tecnica che permette di istanziare ... recupera

omtivazione: es. capire che tipo di istruzione abbiamo davanti $\rightarrow$ il downcasting aiuta a recuperare maggiore informazione dagli iteratori

uso esempio del downcasting dunque: specializzare l'estrazione di informazione durante il pass (?)

\subsection{interfacce dei passi llvm}

llvm fornisce gi\`a interfacce diverse:
\begin{itemize}
  \item basicblockpass: itera su bb
  \item callgraphsccpass: itera sui nodi del cg
  \item functionpass: itera sulla lista di funzioni del modulo
  \item eccetera
\end{itemize}

diversificate appositamente per passi con intenzione diversa: permette di scegliere a che "grana" opera il pass di ottimizzazione (di che livello di informazione ho bisogno? magari non mi serve essere a livello di modulo ma direttamente a livello di ad es. loop)


\subsection{new pass manager}

solitamente ha una pipeline "statica" (predefinita) di passi $\rightarrow$ alterabile invocando una sequenza arbitraria tramite cmd line: \lstinline|opt -passes='pass1,pass2' /tmp/a.ll -S| o \lstinline|opt -p pass1, ...|

\section{esercizio 1 - IR e CFG}

\begin{itemize}
  \item per ognuno dei test benchmarks produrre la IR con clang e analizzarla, cercando di capire cosa significa ogni parte
  \item disegnare il CFG per ogni funzione
\end{itemize}

\begin{emphasize}
    usiamo clang per produrre la IR da dare in pasto al middle-end:

    \lstinline|clang -O2 -emit-llvm -S -c test/Loop.c -o test/Loop.ll|

    oppure prima produco bytecode e poi disassemblo per produrre la forma assembly

    \lstinline|clang -O2 -emit-llvm -c test/Loop.c -o test/Loop.bc|

    \lstinline|llvm-dis test/Loop.bc -o=./test/Loop.ll|
\end{emphasize}

\section{esercizio 2 - TestPass}

in questo corso scriveremo i passi di analisi e ottimizzazione come \textbf{plugin} per il pass manager di llvm - modo valido e conveniente per scrivere passi, evitando di dover ricompilare ogni volta llvm (andremo ad usare l'interfaccia plugin appunto, che ci consente uno sviluppo esterno al build tree di llvm - il compilato viene poi linkato come libreria dinamica)

\begin{emphasize-blue}
   istruzioni :
   \begin{itemize}
     \item crea un workspace con root dir es. \lstinline|mkdir lab_compilatori && export ROOT_LABS=/path/to/lab_compilatori| 
     \item scarica i file di lab 1 in \lstinline|ROOT_LABS/Lab1|
     \item prova a settare l'env e compilare: 
       \begin{lstlisting}
export LLVM_DIR=<installation/dir/of/llvm/19>
mkdir build
cd build
cmake -DLT_LLVM_INSTALL_DIR=dollarsignLLVM_DIR
source/dir/test/pass>/
make\end{lstlisting}
     
   \end{itemize}
\end{emphasize-blue}

\begin{emphasize}
    tua posizione di llvm: \lstinline|/opt/homebrew/opt/llvm|
\end{emphasize}

\begin{emphasize-blue}
    \begin{itemize}
      \item vedi script setup.sh (che prevede cartella build gia makeata in precedenza, altrimenti aggiungi mkdir) per come buildare il passo, o slide 41
      \item inserisci cartella \lstinline|test| con loop e fibonacci
      \item a questo punto invoca l'ottimizzatore \lstinline|opt| con il flag di override del default pass manager:
        \begin{lstlisting}[language=bash]
opt -load-pass-plugin <path/to/build/dir>/LibTestPass.so -passes=test-pass test/Loop.bc -o test/LoopTestPass.bc
# .dylib e non .so se su MacOS
# .ll invece di .bc a seconda di quello su cui sto lavorando\end{lstlisting}
       \begin{itemize}
         \item \lstinline|load-pass-plugin| per caricare il plugin appena buildato
         \item \lstinline|passes=test-pass| oppure \lstinline|-p test-pass| per inserire il nuovo pass da noi creato
         \item al momento non stiamo ottimizzando ma solo analizzando, quindi posso anche sostituire il \lstinline|-o| con \lstinline|-disable-output|
       \end{itemize}
        
    \end{itemize}
    
\end{emphasize-blue}

l'esercizio prevede di estendere il passo TestPass, di modo che analizzi la IR e stampi alcune informazioni utili per ciascuna delle funzioni che compaiono nel programma di test:
\begin{enumerate}
  \item nome
  \item numero argomenti (\lstinline|N+*| in caso di funzione variadica, vedi slide 46)
  \item numero chiamate a funzione nello stesso modulo
  \item numero BB
  \item numero Istruzioni
\end{enumerate}



\end{document}

